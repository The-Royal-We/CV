\cvsection{Projects}
\begin{cventries}
  \cventry
    {AOL Inc.}
    {CICD with Docker, Jenkins and Gradle}
    {Dublin, Ireland}
    {Jul. 2015}
    {
      \begin{cvitems}
        \item {Gave a brown bag discussion on the Continuous Integration and Continuous Deployment system that I developed during my time in AOL. Gave a brief tutorial on Docker then how my system tied into Gradle so a developer can easily integrate it into their own projects. Next I went on to how my system automatically built software using Jenkins and the Git hook plugin. Finally I illustrated how my architecture was able to deploy the application, with their respective configurations, to any environment using Docker.}
      \end{cvitems}
    }
    \cventry
    {Maynooth University}
    {SFARRO: Safe Filesystem via Automatic Remount Read-Only }
    {Maynooth, Ireland}
    {Sept. 2015}
    {
        \begin{cvitems}
        \item {Currently developing an application using Linux OS and the FUSE (Filesystem in Userspace) libraries that would allow a user to input a mountpoint and SFARRO automatically remounts that mountpoint to read-only when no writes have been committed it after either a default or user specified interval of time. The aim of the project is to provide Linux users a safety net with regards to removing their devices from the system. }
        \end{cvitems}
    }
   \cventry
    {Personal}
    {Twitter Sentiment Analysis Tool }
    {}
    {}
    {
        \begin{cvitems}
        \item {This is a personal project that I have been developing on and off for the past couple of months.
        The aim of the project was to determine what was the general sentiment for a given subject. As it stands, the project assigns only the level of emotional intensity to a particular tweet. What I have planned for the future is to instrument this project into a Spring/Hibernate application in order to store these metrics and to utilise a dashboard framework such as d3.js in order to visualise these datapoints. Also, in the near future I'd like to implement  better natural language processing in order to break down these tweets into find the preelevant emotion, rather than determine emotional intensity.}
        \end{cvitems}
    }
\end{cventries}